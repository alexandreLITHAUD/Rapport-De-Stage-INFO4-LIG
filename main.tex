\documentclass[a4paper,french,12pt, titlepage]{article}

%%%% paquetages %%%%
\usepackage[french,english]{babel}  %définition de la langue
\usepackage[T1]{fontenc}
\usepackage[utf8]{inputenc} %définition de l'encodage
\usepackage{fullpage} %pour réduire les marges
\usepackage{graphicx} %pour figures
\usepackage{comment}
\usepackage{xcolor}
\usepackage{listings}
\usepackage{amsmath}
\usepackage{tikz}
\usepackage{hyperref}
\usepackage{wrapfig}
\usepackage{fancyhdr} %headers
%\usepackage{glossaries}
\usepackage{todonotes}
\usepackage[export]{adjustbox}

\usepackage{titling}

\renewcommand{\maketitle}{%
  \maketitlehooka
  \maketitlehookb
  \maketitlehookc
  \maketitlehookd
}

\renewcommand{\maketitlehooka}{% 
    \vspace{-5.5cm}
    
    \noindent
    \raisebox{15ex}{\includegraphics[height=10ex]{logos/logo_polytech_full.png}}
    \hfill\raisebox{16ex}{\includegraphics[height=10ex]{logos/logo_lig_new.png.png}}
    \space\space\space
    \raisebox{12ex}{\includegraphics[height=15ex]{logos/logo_inp_uga.png}}
    
    \vfill
    
    \bigskip
    \begin{center} \large
    Alexandre Lithaud
    
    INFO4 - Polytech Grenoble
    
    Rapport de stage 2022/2023
    
    \vfill
    \begin{Large}
    \textbf{Contribution au projet NixOS Compose}
    \end{Large}
    
    \vfill
    
    Tome Principal
    
    \begin{small}
    ET
    \end{small}
    
    Annexe
    
    \end{center}
}

\renewcommand{\maketitlehookd}{% 
    \vfill{}  \large\par\noindent  
    \begin{center}
    2022/2023\\
    17 Avril 2023 - 28 Juillet 2023
    \end{center}
    \vspace{-0.5cm}
}


\newcommand{\makefooter}{%
  \makefooterhooka
}

\newcommand{\makefooterhooka}{% 
    \begin{center}
        \begin{Large}
        DOS DU RAPPORT
        \end{Large}
    \end{center}
    
    
    %\vspace{-5.5cm}
    %\noindent
    \textbf{Etudiant} : Alexandre Lithaud
    \hfill \textbf{Année d’étude dans la spécialité :}
    
    \hfill INFO4 2022/2023
    
    \hfill
    
    \textbf{Entreprise} : Laboratoire d'informatique de Grenoble 

    \textbf{Adresse complète} : Bâtiment IMAG, 700, AV. Centrale, 38401
Saint Martin d'Hères

    \textbf{Téléphone (standard)} : 07.87.30.90.36
    
    \hfill
    
    \textbf{Responsable administratif} : Noel de Palma 

    \textbf{Téléphone} : 04.57.42.14.78

    \textbf{Courriel} : noel.de-palma@univ-grenoble-alpes.fr

    \hfill
    
    \textbf{Tuteur de stage (organisme d’accueil)} : Olivier Richard

    \textbf{Téléphone} : 06.32.29.09.18

    \textbf{Courriel} : olivier.richard@imag.fr
    
    \hfill
    
    \textbf{Enseignant-référent} : Nicolas Palix

    \textbf{Téléphone} : 04.57.42.15.38 

    \textbf{Courriel} : nicolas.palix@imag.fr 

    \hfill
    
    \textbf{Titre} : Contribution au projet NixOS Compose
    
    \hfill

    \textbf{Résumé} : Mon stage de 15 semaines au Laboratoire
d'Informatique de Grenoble (LIG) a été axé sur l'utilisation de Nix et
NixOS, ainsi que sur la maintenance et l'amélioration de logiciels de
recherche tels que OAR et EAR. J'ai également travaillé avec le système
Grid5000, une infrastructure distribuée pour la recherche.\newline

En utilisant NixOS-compose, j'ai pu déployer et gérer des environnements
de développement reproductibles. J'ai acquis une solide compréhension de
la gestion des paquets et des environnements isolés. J'ai contribué à la
maintenance de logiciels tels que OAR et EAR, améliorant leur stabilité
et leurs performances. J'ai également eu l'opportunité de travailler
avec le système Grid5000, ce qui m'a permis de tester mes développements
dans un environnement réel.\newline

Ce stage m'a permis d'acquérir des compétences techniques avancées et de
développer des compétences en gestion de projet et en collaboration
d'équipe. J'ai apprécié les défis liés à la recherche et au
développement de logiciels dans un environnement exigeant.\newline

Dans l'ensemble, ce stage a été une expérience enrichissante qui a
renforcé ma passion pour l'informatique et ouvert de nouvelles
perspectives pour ma carrière future dans la recherche et le
développement de logiciels avancés.
}

\title{\Huge \bfseries\begin{center}Contribution au projet NixOS
Compose\end{center}}
\author{Alexandre Lithaud}
\date{17 Avril 2023 - 28 Juillet 2023}

\lstset{language=C++,
                basicstyle=\ttfamily,
                keywordstyle=\color{blue}\ttfamily,
                stringstyle=\color{red}\ttfamily,
                showstringspaces=false,
                %commentstyle=\color{magenta}\ttfamily,
                morecomment=[l][\color{magenta}]{\#}
}

\newcommand*\xor{\mathbin{\oplus}}
\newcommand{\paragraphnewline}[1]{\hypertarget{par#1}{\paragraph{#1}\mbox{}}}


\pagestyle{fancy}
\fancyhf{}
\rhead{Alexandre Lithaud - 2022/2023}
\lhead{Rapport de stage}
\rfoot{Page \thepage}
\renewcommand{\headrulewidth}{1pt}
\renewcommand{\footrulewidth}{1pt}
\setlength{\headheight}{15pt}
\headsep = 1.0cm

% references

\usepackage{csquotes}
\usepackage{biblatex}
\bibliography{references.bib}


% \makenoidxglossaries

\begin{document}
\selectlanguage{french}

\begin{titlingpage}
\maketitle
\end{titlingpage}

\begin{center}
    \item \paragraphnewline{Remerciements}
\end{center}

Je tiens à tout d'abord à remercier le Laboratoire Informatique de
Grenoble et tous ces membres pour l'accueil chaleureux que j'ai reçu à
mon arrivée au laboratoire ainsi que pour l'ambiance générale du stage
qui a été exemplaire.\newline

Je remercie également Monsieur Olivier Richard et Monsieur Nicolas
Palix, respectivement mon tuteur et mon référent de stage pour leurs
conseils ainsi que leurs pédagogies qui m'ont permis de réaliser mes
missions dans les meilleures conditions possibles et de grandement
monter en compétence durant ce stage.\newline

Enfin, je suis reconnaissant envers Quentin GUILLOTEAU et Adrien FAURE,
respectivement doctorant et chercheur au Laboratoire Informatique de
Grenoble pour les inestimables conseils et les réponses dispensés lors
de mes différentes missions.

\newpage

\selectlanguage{french}
\begin{center}
    \item \paragraphnewline{Résumé}
\end{center}

Mon stage de 15 semaines au Laboratoire d'Informatique de Grenoble (LIG)
a été axé sur l'utilisation de Nix et NixOS, ainsi que sur la
maintenance et l'amélioration de logiciels de recherche tels que OAR et
EAR. J'ai également travaillé avec le système Grid5000, une
infrastructure distribuée pour la recherche.\newline

En utilisant NixOS-compose, j'ai pu déployer et gérer des environnements
de développement reproductibles. J'ai acquis une solide compréhension de
la gestion des paquets et des environnements isolés. J'ai contribué à la
maintenance de logiciels tels que OAR et EAR, améliorant leur stabilité
et leurs performances. J'ai également eu l'opportunité de travailler
avec le système Grid5000, ce qui m'a permis de tester mes développements
dans un environnement réel.\newline

Ce stage m'a permis d'acquérir des compétences techniques avancées et de
développer des compétences en gestion de projet et en collaboration
d'équipe. J'ai apprécié les défis liés à la recherche et au
développement de logiciels dans un environnement exigeant.\newline

Dans l'ensemble, ce stage a été une expérience enrichissante qui a
renforcé ma passion pour l'informatique et ouvert de nouvelles
perspectives pour ma carrière future dans la recherche et le
développement de logiciels avancés.


\textbf{\textit{mots-clés---}} Nix, Programmation Fonctionnel, NixOS,
NixOS-Compose, Grid5000, Systèmes de fichiers, Logiciel de Recherche,
Maintenance, HPC, Infrastructure Distribué.

\selectlanguage{english}
\begin{center}
    \item \paragraphnewline{Abstract}
\end{center}

My 15-week placement at the Laboratoire d'Informatique de Grenoble (LIG)
focused on using Nix and NixOS, as well as maintaining and improving
research software such as OAR and EAR. I also worked with the Grid5000
system, a distributed infrastructure for research.\newline

Using NixOS-compose, I was able to deploy and manage reproducible
development environments. I gained a solid understanding of package
management and isolated environments. I helped maintain software such as
OAR and EAR, improving their stability and performance. I also had the
opportunity to work with the Grid5000 system, which enabled me to test
my developments in a real environment.\newline

This placement has enabled me to acquire advanced technical skills and
to develop project management and teamwork skills. I enjoyed the
challenges of researching and developing software in a demanding
environment.\newline

Overall, this internship has been a rewarding experience that has
strengthened my passion for computer science and opened up new prospects
for my future career in advanced software research and development.

\textbf{\textit{Keywords---}} Nix, Fonctional Programming, NixOS,
NixOS-Compose, Grid5000, File Systems, Research Softwares, Maintenance,
HPC, Distributed Infrastructure

\selectlanguage{french}
\newpage

\tableofcontents
\newpage

\hypertarget{introduction}{%
\section{Introduction}\label{introduction}}

Ce rapport va représenter mon expérience de stage au Laboratoire
Informatique de Grenoble. Mon stage de 15 semaines à débuter le 17 avril
2023. Au cours de cette période j'ai eu l'opportunité de travailler sur
divers projet informatiques en lien avec les technologies de Nix, NixOS
et le HPC (\emph{High performance computing}). Ainsi que sur la
maintenance et l'amélioration de logiciel et recherche tels que OAR et
EAR. Cette opportunité m'a donné l'occasion de travailler avec le
système Grid5000, qui offre une infrastructure distribuée pour
l'exécution de travaux de recherche à grande échelle.\newline

L'objectif principal de mon stage était, en premier lieu, de contribuer
au projet au projet NixOS-Compose, un outils puissant qui facilite le
déploiement et la gestion d'environnement de développement reproductible
spécialisé pour le HPC en déployant directement plusieurs machines sur
Grid5000 à la manière de Docker Compose (d'où le nom). Afin de pourvoir
réaliser cette tache il était important de monter en compétences sur
Nix, NixOS. Grâce à cette expérience, j'ai pu approfondir ma
compréhension des principes fondamentaux de la gestion des paquets et
des environnements isolés, la configuration de système basé NixOS, le
paradigme de programmation fonctionnelle ainsi que le déploiement
d'application fonctionnelle dans un environnement d'HPC.\newline

En parallèle, j'ai participé à la maintenance et à l'amélioration de
logiciel de recherche tels que OAR et EAR en les mettant à jour avec la
dernière version de Nix par exemple. OAR joue un role crucial dans la
planification de travaux de recherche sur des infrastructures distribué
comme Grid5000 notamment. EAR quant à lui, permet d'instrumenter et donc
de quantifié les performances d'applications distribuées.J'ai pu
contribuer à l'amélioration de leur stabilité, de leurs performances et
de leurs fonctionnalités, en collaborant étroitement avec l'équipe de
développement du laboratoire.\newline

De plus, j'ai eu l'opportunité de travailler en utilisant le système
Grid5000, qui m'a permis de déployer et de tester mes compositions
directement dans un environnement réel et reproductible. Cette
expérience m'a offert une compréhension bien plus poussé sur les
méthodes de déploiement de logiciel, à l'importance de d'évolutivité et
à la gestion des ressources et à la fiabilité des systèmes
distribués.\newline

Dans ce rapport, je décrirai en détail les différentes tâches et projets
auxquels j'ai participé tout au long de mon stage, en mettant l'accent
sur les compétences acquises, les résultats obtenus et les leçons
apprises. Je présenterai également une analyse critique de mes
réalisations, ainsi que des suggestions pour des améliorations futures.
Ce rapport témoigne de ma progression en tant que professionnel de
l'informatique et des contributions significatives que j'ai apportées au
sein du LIG.\newline

Test de citation \cite{grid5000} je fait des test ici\newline

\newpage

\hypertarget{test-de-mise-en-page}{%
\section{Test de mise en page}\label{test-de-mise-en-page}}

\newpage

\hypertarget{annexe}{%
\section{Annexe}\label{annexe}}

\listoffigures

\printbibliography

\begin{titlingpage}
\newpage
\makefooter
\end{titlingpage}

\end{document}
