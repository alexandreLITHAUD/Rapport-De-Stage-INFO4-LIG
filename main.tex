\documentclass[a4paper,french,12pt, titlepage]{article}

%%%% paquetages %%%%
\usepackage[french,english]{babel}  %définition de la langue
\usepackage[T1]{fontenc}
\usepackage[utf8]{inputenc} %définition de l'encodage
\usepackage{fullpage} %pour réduire les marges
\usepackage{graphicx} %pour figures
\usepackage{comment}
\usepackage{xcolor}
\usepackage{listings}
\usepackage{amsmath}
\usepackage{tikz}
\usepackage{hyperref}
\usepackage{wrapfig}
\usepackage{fancyhdr} %headers
\usepackage{glossaries}
\usepackage{todonotes}
\usepackage[export]{adjustbox}

\usepackage{titling}

\makeglossaries

\newglossaryentry{grid5000}
{
    name=Grid5000,
    description={Réseaux de noeuds hébergé un peux partout en France destiné à réalisé des essais pour la recherche dans le domaine de l'informatique distribué et parallèle.}
}

\newglossaryentry{hpc}
{
    name=HPC,
    description={Branche de l'informatique qui cherche à traiter des données et à effectuer des calculs complexe à grande vitesse.}
}

\newglossaryentry{oar}
{
    name=OAR,
    description={Logiciel d’ordonnancement de ressources informatique.}
}

\newglossaryentry{reproductibilité}
{
    name=Reproductibilité,
    description={Qualité d'une mesure qui donne les mêmes résultats si on la répète dans des conditions différentes et à des époques différentes.}
}

\newglossaryentry{nix}
{
    name=Nix,
    description={Gestionnaire de paquet fonctionnel et language de programmation fonctionnel permettant la description de paquet.}
}

\newglossaryentry{nixos}
{
    name=NixOS,
    description={Système d'exploitation utilisant Nix, son langage et son gestionnaire de paquet.}
}

\newglossaryentry{nix-store}
{
    name=Nix-store,
    description={Système de gestion des paquet dans Nix}
}

\newglossaryentry{nixos-compose}
{
    name=NixOS-Compose,
    description={Logiciel permettant de créer et déployer des infrastructures distribué simplement et de manière reproductible en Nix}
}

\newglossaryentry{composition}
{
    name=Composition,
    description={Description d'une infrastructure distribué en Nix, compilable et deployable par NixOS-Compose}
}

\newglossaryentry{depedency-hell}
{
    name=Depedency-Hell,
    description={Terme familier désignant le problème ou des applications dépendent de certaines version spécifique d'application, bloquant donc le système.}
}

\newglossaryentry{nixpkgs}
{
    name=Nixpkgs,
    description={Dépôt principal de paquet dans l'environnement Nix}
}

\newglossaryentry{nur}
{
    name=NUR,
    description={Dépôt supplémentaire de paquet Nix permettant le partage simple de dérivations}
}

\newglossaryentry{garbage-collection}
{
    name=Garbage-Collection,
    description={Processus consistant à faire de la place dans la mémoire d'un ordinateur en supprimant les données qui ne sont plus nécessaires ou utilisées.}
}

\newglossaryentry{pinning}
{
    name=Pinning,
    description={Processus de fixation des versions des ressources externes, telles que les paquets ou les dépendances à des versions ou révisions spécifiques.}
}

\renewcommand{\maketitle}{%
  \maketitlehooka
  \maketitlehookb
  \maketitlehookc
  \maketitlehookd
}

\renewcommand{\maketitlehooka}{% 
    \vspace{-5.5cm}
    
    \noindent
    \raisebox{15ex}{\includegraphics[height=10ex]{logos/logo_polytech_full.png}}
    \hfill\raisebox{16ex}{\includegraphics[height=10ex]{logos/logo_lig_new.png.png}}
    \space\space\space
    \raisebox{12ex}{\includegraphics[height=15ex]{logos/logo_inp_uga.png}}
    
    \vfill
    
    \bigskip
    \begin{center} \large
    Alexandre Lithaud
    
    INFO4 - Polytech Grenoble
    
    Rapport de stage 2022/2023
    
    \vfill
    \begin{Large}
    \textbf{Contribution au projet NixOS Compose}
    \end{Large}
    
    \vfill
    
    Tome Principal
    
    \begin{small}
    ET
    \end{small}
    
    Annexe
    
    \end{center}
}

\renewcommand{\maketitlehookd}{% 
    \vfill{}  \large\par\noindent  
    \begin{center}
    2022/2023\\
    17 Avril 2023 - 28 Juillet 2023
    \end{center}
    \vspace{-0.5cm}
}


\newcommand{\makefooter}{%
  \makefooterhooka
}

\newcommand{\makefooterhooka}{% 
    \begin{center}
        \begin{Large}
        DOS DU RAPPORT
        \end{Large}
    \end{center}
    
    
    %\vspace{-5.5cm}
    %\noindent
    \textbf{Etudiant} : Alexandre Lithaud
    \hfill \textbf{Année d’étude dans la spécialité :}
    
    \hfill INFO4 2022/2023
    
    \hfill
    
    \textbf{Entreprise} : Laboratoire d'informatique de Grenoble 

    \textbf{Adresse complète} : Bâtiment IMAG, 700, AV. Centrale, 38401
Saint Martin d'Hères

    \textbf{Téléphone (standard)} : 07.87.30.90.36
    
    \hfill
    
    \textbf{Responsable administratif} : Noel de Palma 

    \textbf{Téléphone} : 04.57.42.14.78

    \textbf{Courriel} : noel.de-palma@univ-grenoble-alpes.fr

    \hfill
    
    \textbf{Tuteur de stage (organisme d’accueil)} : Olivier Richard

    \textbf{Téléphone} : 06.32.29.09.18

    \textbf{Courriel} : olivier.richard@imag.fr
    
    \hfill
    
    \textbf{Enseignant-référent} : Nicolas Palix

    \textbf{Téléphone} : 04.57.42.15.38 

    \textbf{Courriel} : nicolas.palix@imag.fr 

    \hfill
    
    \textbf{Titre} : Contribution au projet NixOS Compose
    
    \hfill

    \textbf{Résumé} : En 4ème année d'ingénieur en informatique, j'ai eu
l'opportunité de faire un stage de 15 semaines au Laboratoire
Informatique de Grenoble (LIG), au sein de l'équipe DATAMOVE.\newline

Durant ce stage, j'ai eu comme objectif d'utiliser et d'améliorer
l'outil NixOS-Compose, ainsi que de créer différentes compositions dans
l'optique de les utiliser à une fin de recherche. NixOS-Compose (ou NXC)
est un logiciel créé par l'équipe, permettant de décrire une
infrastructure complexe de plusieurs nœuds, en mettant l'accent sur la
reproductibilité et la simplicité de mise en place. De plus, j'ai été
amené à contribuer à la maintenance de logiciels tels que OAR et EAR,
améliorant leur stabilité par le biais de mise à jour. Le tout en
utilisant le système Grid5000 qui m'a permis de tester mes
développements dans un environnement réel.\newline

Durant ce rapport, vous allez suivre la création des différentes
compositions que j'ai créée dans le but de tester les performances de
plusieurs systèmes de fichiers distribués dans le réseau de nœud
Grid5000.
}

\title{\Huge \bfseries\begin{center}Contribution au projet NixOS
Compose\end{center}}
\author{Alexandre Lithaud}
\date{17 Avril 2023 - 28 Juillet 2023}

\lstset{language=C++,
                basicstyle=\ttfamily,
                keywordstyle=\color{blue}\ttfamily,
                stringstyle=\color{red}\ttfamily,
                showstringspaces=false,
                %commentstyle=\color{magenta}\ttfamily,
                morecomment=[l][\color{magenta}]{\#}
}

\newcommand*\xor{\mathbin{\oplus}}
\newcommand{\paragraphnewline}[1]{\hypertarget{par#1}{\paragraph{#1}\mbox{}}}


\pagestyle{fancy}
\fancyhf{}
\rhead{Alexandre Lithaud - 2022/2023}
\lhead{Rapport de stage}
\rfoot{Page \thepage}
\renewcommand{\headrulewidth}{1pt}
\renewcommand{\footrulewidth}{1pt}
\setlength{\headheight}{15pt}
\headsep = 1.0cm

% references

\usepackage{csquotes}
\usepackage{biblatex}
\bibliography{references.bib}


% \makenoidxglossaries

\begin{document}
\selectlanguage{french}

\begin{titlingpage}
\maketitle
\end{titlingpage}

\begin{center}
    \item \paragraphnewline{Remerciements}
\end{center}

Je tiens à tout d'abord à remercier le Laboratoire Informatique de
Grenoble et tous ces membres pour l'accueil chaleureux que j'ai reçu à
mon arrivée au laboratoire ainsi que pour l'ambiance générale du stage
qui a été exemplaire.\newline

Je remercie également Monsieur Olivier Richard et Monsieur Nicolas
Palix, respectivement mon tuteur et mon référent de stage pour leurs
conseils ainsi que leurs pédagogies qui m'ont permis de réaliser mes
missions dans les meilleures conditions possibles et de grandement
monter en compétence durant ce stage.\newline

Je tiens aussi à remercier Pierre NEYRON, pour toute l'aide que j'ai
reçu et pour les explications avancé sur le fonctionnement de
Grid5000.\newline

Enfin, je suis reconnaissant envers Quentin GUILLOTEAU et Adrien FAURE,
respectivement doctorant et chercheur au Laboratoire Informatique de
Grenoble pour les inestimables conseils et les réponses dispensés lors
de mes différentes missions.

\newpage

\selectlanguage{french}
\begin{center}
    \item \paragraphnewline{Résumé}
\end{center}

En 4ème année d'ingénieur en informatique, j'ai eu l'opportunité de
faire un stage de 15 semaines au Laboratoire Informatique de Grenoble
(LIG), au sein de l'équipe DATAMOVE.\newline

Durant ce stage, j'ai eu comme objectif d'utiliser et d'améliorer
l'outil NixOS-Compose, ainsi que de créer différentes compositions dans
l'optique de les utiliser à une fin de recherche. NixOS-Compose (ou NXC)
est un logiciel créé par l'équipe, permettant de décrire une
infrastructure complexe de plusieurs nœuds, en mettant l'accent sur la
reproductibilité et la simplicité de mise en place. De plus, j'ai été
amené à contribuer à la maintenance de logiciels tels que OAR et EAR,
améliorant leur stabilité par le biais de mise à jour. Le tout en
utilisant le système Grid5000 qui m'a permis de tester mes
développements dans un environnement réel.\newline

Durant ce rapport, vous allez suivre la création des différentes
compositions que j'ai créée dans le but de tester les performances de
plusieurs systèmes de fichiers distribués dans le réseau de nœud
Grid5000.


\textbf{\textit{mots-clés---}} Nix, Reproductibilité, Programmation
Fonctionnel, Laboratoire, NixOS, NixOS-Compose, Grid5000, Systèmes de
fichiers, Logiciel de Recherche, Maintenance, HPC, Infrastructure
Distribué.

\selectlanguage{english}
\begin{center}
    \item \paragraphnewline{Abstract}
\end{center}

In my 4th year as a computer science engineer, I had the opportunity to
do a 15-week internship at the IT Laboratory of Grenoble (LIG), in the
DATAMOVE team.\newline

During this placement, my aim was to use and improve the NixOS-Compose
tool, and to create various compositions with a view to using them for
research purposes. NixOS-Compose (or NXC) is a piece of software created
by the team, enabling a complex infrastructure of several nodes to be
described, with the emphasis on reproducibility and simplicity of
implementation. I also contributed to the maintenance of software such
as OAR and EAR, improving their stability through updates. All this was
done using the Grid5000 system, which enabled me to test my developments
in a real environment.\newline 

In this report, you will follow the creation of the various compositions
I created in order to test the performance of several distributed file
systems in the Grid5000 node network.

\textbf{\textit{Keywords---}} Nix, Reproductibility, Functional
Programming, Laboratory, NixOS, NixOS-Compose, Grid5000, File Systems,
Research Softwares, Maintenance, HPC, Distributed Infrastructure

\selectlanguage{french}
\newpage

\tableofcontents
\newpage

\listoffigures

\newpage

\hypertarget{introduction}{%
\section{Introduction}\label{introduction}}

Ce rapport va représenter mon expérience de stage au Laboratoire
Informatique de Grenoble. Mon stage de 15 semaines à débuter le 17 avril
2023. Au cours de cette période j'ai eu l'opportunité de travailler sur
divers projet informatiques en lien avec les technologies de \Gls{nix},
\Gls{nixos} et le \Gls{hpc} (\emph{High performance computing}). Ainsi
que sur la maintenance et l'amélioration de logiciel et recherche tels
que \Gls{oar} et EAR. Cette opportunité m'a donné l'occasion de
travailler avec le système \Gls{grid5000}, qui offre une infrastructure
distribuée pour l'exécution de travaux de recherche à grande
échelle.\newline

L'objectif principal de mon stage était, en premier lieu, de contribuer
au projet \Gls{nixos-compose}, un outils puissant qui facilite le
déploiement et la gestion d'environnement de développement reproductible
spécialisé pour le HPC en déployant directement plusieurs machines sur
Grid5000 à la manière de Docker Compose. Afin de pourvoir réaliser cette
tache il était important de monter en compétences sur le gestionnaire de
paquet fonctionnel Nix et NixOS. Grâce à cette expérience, j'ai pu
approfondir ma compréhension des principes fondamentaux de la gestion
des paquets et des environnements isolés, la configuration de système
basé NixOS, le paradigme de programmation fonctionnelle ainsi que le
déploiement d'application fonctionnelle dans un environnement
d'HPC.\newline

En parallèle, j'ai participé à la maintenance et à l'amélioration de
logiciel de recherche tels que OAR et EAR en les mettant à jour avec la
dernière version de Nix par exemple. OAR joue un role crucial dans la
planification de travaux de recherche sur des infrastructures distribué
comme Grid5000 notamment. EAR quant à lui, permet d'instrumenter et donc
de quantifié les performances d'applications distribuées.J'ai pu
contribuer à l'amélioration de leur stabilité, de leurs performances et
de leurs fonctionnalités, en collaborant étroitement avec l'équipe de
développement du laboratoire.\newline

De plus, j'ai eu l'opportunité de travailler en utilisant le système
Grid5000, qui m'a permis de déployer et de tester mes
\glspl{composition},c'est-à-dire des descriptions de système distribué
fait en Nix et ce directement dans un environnement réel et
reproductible. Cette expérience m'a offert une compréhension bien plus
poussé sur les méthodes de déploiement de logiciel, à l'importance de
d'évolutivité et à la gestion des ressources et à la fiabilité des
systèmes distribués.\newline

Dans ce rapport, je décrirai en détail les différentes tâches et projets
auxquels j'ai participé tout au long de mon stage, en mettant l'accent
sur les compétences acquises, les résultats obtenus et les leçons
apprises. Je présenterai également une analyse critique de mes
réalisations, ainsi que des suggestions pour des améliorations futures.
Ce rapport témoigne de ma progression en tant que professionnel de
l'informatique et des contributions significatives que j'ai apportées au
sein du LIG.

\newpage

\hypertarget{contexte-du-stage}{%
\section{Contexte du stage}\label{contexte-du-stage}}

\hypertarget{le-laboratoire-informatique-de-grenoble}{%
\subsection{Le Laboratoire Informatique de
Grenoble}\label{le-laboratoire-informatique-de-grenoble}}

\begin{figure}[h]
\centering
\includegraphics[width=0.8\textwidth,height=0.8\textheight,keepaspectratio]{images/imag.png}
\caption{bâtiment IMAG}
\end{figure}

Mon stage s'est déroulé au LIG ou laboratoire informatique de Grenoble,
ce laboratoire ainsi que certains autres sont situés dans le bâtiment
IMAG, situé au centre de Saint-martin-d'Heres. Il est le réceptacle de
nombreux projets de recherches et de recherche. Durant mon temps au LIG,
j'ai eu la possibilité de rencontrer de nombreux professionnels,
représentant des différents laboratoires présent dans le
bâtiment.\newline

Le bâtiment est organisé de la sorte :

\begin{itemize}
\item
  1er étage : AMIES, LJK, MAIMOSINE : \textbf{Mathématique}
\item
  2ème étage : GRICAD, LIG, VERIMAG : \textbf{Informatique}
\item
  3ème et 4ème étages: LIG : \textbf{Informatiques}\newline
\end{itemize}

Durant mon stage j'ai eu l'occasion d'assister à de nombreuses
conférences réalisées par des professionnels du sujet, comme une
conférence sur les FPGA ou sur les stratégies de test dans le monde du
HPC. J'ai aussi eu la chance d'animer un cours d'informatique débranché
destiné à deux classes de seconde, afin de les faire réfléchir sur des
problématiques d'informatique sans l'interférence d'un
ordinateur.\newline

En outre, mon stage au laboratoire ma permis de faire de nombreuses
découvertes et expériences en plus de toutes les connaissances que j'ai
pu accumuler.

\hypertarget{luxe9quipe-datamove}{%
\subsection{L'équipe DATAMOVE}\label{luxe9quipe-datamove}}

L'équipe de recherche DATAMOVE du Laboratoire d'Informatique de Grenoble
(LIG) se consacre à l'étude et au développement de techniques innovantes
dans le domaine du traitement et de la gestion des données. Leur
objectif est de relever les défis liés à la croissance exponentielle des
données et de proposer des solutions efficaces pour leur manipulation,
leur analyse et leur exploitation.\newline

L'équipe est spécialisée dans les piles logicielles distribuées et
l'ordonnancement, généralement dans un environnement de High Performance
Computing. Dans ce laboratoire, le sujet de la \gls{reproductibilité}
est majeur grâce à la complexité des piles logiciels créées.\newline

La reproductibilité est une notion essentielle en recherche, en effet
cela consiste à pouvoir réaliser une expérience à l'identique de la
version d'origine afin d'obtenir le même résultat. Cette approche permet
de garantir l'intégrité et la crédibilité des résultats scientifique. Il
n'est cependant pas aisé de rendre une expérience reproductible en
informatique à cause de l'omniprésence d'états qui peuvent être changés
d'une exécution à une autre. De plus, il peut y avoir des problèmes de
version de logiciel, disparition de ressource, d'accès au ressources de
calcul ou encore la présence d'une variable aléatoire. Tous ces
problèmes, engendre un problème de reproductibilité des
logiciels.\newline

Dans le domaine du \Gls{hpc}, la reproductibilité présente plusieurs
avantages. Tout d'abord, elle permet de valider les méthodes de
modélisation et de simulation, garantissant ainsi que les résultats
obtenus sont fiables et précis, même si il peux être incorrect. Cela
renforce la confiance dans les résultats de recherche et facilite la
collaboration et la comparaison des résultats entre différents
chercheurs et laboratoires. De plus, dans ce genre d'environnement ou
les chercheurs déploie des simulations complexes avec des quantités
massives de données à analyser, il est essentiel que ces résultats
puissent être déterministes, ne serait-ce que pour pouvoir assurer de la
rigueur de la recherche.\newline

C'est dans cette optique que des outils de mise en place de pile
logicielle comme \Gls{nixos-compose} ont été mis en place dans
l'équipe.\newline

\textbf{L'équipe}\newline

En ce jour l'équipe DATAMOVE est composé de 34 personnes :

\begin{itemize}
\item
  10 chercheurs
\item
  16 étudiants en thèse
\item
  5 ingénieurs
\item
  3 assistants\newline
\end{itemize}

Cette équipe est dirigée par Monsieur Bruno RAFFIN. Olivier RICHARD,
Quentin QUILLOTEAU et Adrien FAURE sont tous membres de cette équipe.
Respectivement en tant que chercheur, étudiant en thèse et
ingénieur.\newline

\textbf{Quelques projets phares de l'équipe}\newline

\Gls{oar} est un gestionnaire de ressources distribuées conçu pour les
environnements de calcul intensif. Il permet aux chercheurs de
planifier, de contrôler, répondre et alloué des ressources demandé par
un utilisateur dans des environnements telles que les clusters de
calcul, les grilles de calcul et les infrastructures de cloud computing.
OAR offre une gestion fine des tâches, des files d'attente et des
politiques de priorité, permettant ainsi une utilisation efficace et
équitable des ressources. Cet outil facilite la planification des
travaux de recherche et optimise l'utilisation des infrastructures
informatiques. Cet outil est notamment utilisé dans Grid5000 pour la
réservation et l'allocation des ressources.\newline

Melissa, quant à lui, est un framework pour le développement
d'applications parallèles et distribuées. Il fournit une infrastructure
logicielle permettant aux chercheurs de concevoir et d'exécuter des
applications haute performance sur des environnements hétérogènes et
distribués. Melissa simplifie le processus de développement en
fournissant des abstractions de haut niveau pour la programmation
parallèle, l'orchestration des tâches et la gestion des données
distribuées. Cet outil permet aux chercheurs de tirer pleinement parti
des ressources informatiques disponibles et de développer des
applications performantes et évolutives.\newline

NixOS-Compose est un outil conçu pour les expériences dans les systèmes
distribués. Il génère des environnements distribués reproductibles afin
d'être déployés sur une plateforme physique ou virtualisé. C'est le
logiciel auquel j'ai le principalement contribué et utilisé lors de ce
stage.

\newpage

\hypertarget{missions-au-sein-de-luxe9quipe-datamove}{%
\section{Missions au sein de l'équipe
DATAMOVE}\label{missions-au-sein-de-luxe9quipe-datamove}}

\hypertarget{lenvironnement-nix-et-nixos}{%
\subsection{L'environnement Nix et
NixOS}\label{lenvironnement-nix-et-nixos}}

\hypertarget{nix}{%
\subsubsection{Nix}\label{nix}}

\begin{figure}[h]
\centering
\includegraphics[width=0.2\textwidth,height=0.2\textheight,keepaspectratio]{logos/nixlogo.png}
\caption{Logo Nix}
\end{figure}

Nix a été la technologie clé de mon stage. C'est la technologie phare
que j'ai été amené à étudier et à comprendre tout au long de mon stage
au Laboratoire Informatique de Grenoble.\newline

Nix est un gestionnaire de paquet fonctionnel et un outil de déploiement
d'environnement reproductible. Il permet la gestion des dépendances
logicielles de manière déclarative et garantit la reproductibilité des
environnements de développement, et ce, en utilisant son propre langage,
le \emph{Nix Expression Language}, communément appelé Nix. Le langage
Nix est fonctionnel, pure à évaluation paresseuse. Comme dit
précédemment, le mot clé de Nix est reproductibilité. Son langage, sa
gestion des paquets et son architecture fonctionnelle lui permettent
d'obtenir un résultat toujours identique pour des conditions
identiques.\newline

\begin{figure}[h]
\centering
\includegraphics[width=0.8\textwidth,height=0.8\textheight,keepaspectratio]{images/codebasenixdark.png}
\caption{Code nix basique}
\end{figure}

La particularité de Nix réside dans son approche fonctionnelle. En
effet, Nix ne dépend pas de l'installation globale des paquets dans le
système d'exploitation. À la place, chaque paquet est traité comme une
fonction pure qui prend en entrée une version spécifique du paquet et de
ses dépendances et retourne en sortie une version spécifique du paquet.
Grâce à ce système, on met de côté le problème de \Gls{depedency-hell}
si commun dans la plupart des gestionnaires de paquet et l'on s'assure
que chaque paquet ait la version requise et demandée.\newline

Enfin, Nix est aussi capable de générer des environnements isolés
configurables. Il est possible de créer des environnements shell
possédant des dépendances spécifiques. Cela évite les conflits entre les
différentes versions d'un paquet utilisé par des applications, mais
aussi, permet de faciliter la portabilité, car une dépendance peut être
utilisée sans avoir été installé par l'utilisateur (c'est ce que j'ai
fait pour compiler ce rapport par exemple !).\newline

\textbf{Le Nix-Store}\newline

Le store Nix est un composant essentiel pour assurer le bon
fonctionnement et la reproductibilité du système de gestion de paquet
Nix. Il fonctionne sous forme de système de fichier hiérarchique qui
stocke tous les paquets présents dans la machine dans un dossier
spécifique nommé store. La gestion diffère donc des gestionnaires de
paquet classique comme apt ou pacman qui stocke tout en utilisant un
système de fichier standard.\newline

Le store Nix repose sur 5 principes clés :

\begin{itemize}
\item
  \textbf{Hashing des paquets} : Chaque paquet ou dépendances dans le
  store est identifié par un hachage spécifique parfait basé sur son
  contenu. Grâce à ce système, deux paquets identiques ne seront stockés
  qu'une seule fois. De plus, il est donc possible de stocker plusieurs
  versions d'un même paquet.
\item
  \textbf{Immutabilité des fichiers} : Les paquets présents dans le
  store sont immuables. Il est impossible d'en effectuer une
  modification après leur création. C'est un avantage considérable, car
  cela assure l'intégrité des paquets et limite les effets de bord
  néfaste.
\item
  \textbf{Liens symboliques} : Les fichiers, dossier et dérivations
  présent dans le store sont référencés par des liens symboliques,
  permettant au utilisateur de pouvoir utiliser les paquets présent dans
  le store sans avoir besoin de mettre à jour le PATH ou de connaitre le
  chemin exact (et donc le hash) du paquet.
\item
  \textbf{Gestion des dépendances} : Les paquets présents dans le store
  utilisent des liens symboliques référençant chaque dépendance qu'il
  possède. Cela permet de nous assurer que chaque paquet utilise la
  bonne version de chaque dépendance.
\item
  \textbf{Garbage Collection} : Enfin, le store possède un système de
  \gls{garbage-collection} basé sur des \emph{Garbage root}.
  C'est-à-dire que les paquets installés et donc devant être gardé sont
  stockées en tant que garbage root. À la garbage collection
  (\texttt{nix-collect-garbage}) les garbage root et leurs dépendances
  sont gardés et le reste est élagué par le système. Ce système est
  important, car chaque dépendance est téléchargée et stockée dans le
  store. Ce qui peut rendre le store très lourd.\newline
\end{itemize}

Tous ces principes permettent la reproductibilité des environnements de
développement ainsi que des paquets et application du système. On
s'assure donc une cohérence générale et une prédictibilité du
système.\newline

\begin{figure}[h]
\centering
\includegraphics[width=0.8\textwidth,height=0.8\textheight,keepaspectratio]{images/store.png}
\caption{Exemple d'architecture de store multi user}
\end{figure}

Comme on peut le voir dans la figure 4, il est donc possible avec ce
système de posséder plusieurs versions d'un même paquet. De plus, avec
le système de profil Nix, il est possible de définir quel utilisateur
utilisent quel paquet et donc séparer les utilisateurs. Cependant, peu
importe le nombre d'usagés de la machine, il n'y aura toujours qu'un
seul store global.\newline

\textbf{Les Nix Flakes}\newline

Les flakes sont une fonctionnalité encore expérimentale de Nix qui vise
à d'autant plus améliorer la reproductibilité, la modularité et la
gestion des dépendances dans Nix. Il permet de définir une interface
commune pour importation de ressource extérieure. Bien que toujours en
phase expérimentale, les flakes sont massivement utilisés par la
communauté grâce aux ajouts importants qu'il permet. Ils sont
régulièrement considérés par la communauté des utilisateurs de nix comme
un ajout essentiel au bon fonctionnement actuel de nix et à sa
prospérité.\newline

Afin de réaliser un flake, il suffit de créer un fichier
\texttt{flake.nix}. Un flake ne prend pas de paramètre d'entrée comme
pourrait le faire un script nix classique. À la place, il récupère des
ressources sous forme d'input et les utilisent pour y créer une sortie.
Ces paramètres d'entrées peuvent être un dépôt distant Git ou un autre
flake par exemple.\newline

Comme il ne prend pas de paramètre d'entrée, il ne dépend aucunement de
la configuration de la machine actuelle. À la compilation, un flake crée
un fichier \texttt{flake.lock} qui définie les versions, le type du
dépôt, la dernière date de modification, etc. Ce fichier permet donc
d'avoir une trace des versions utilisées et de pouvoir les réutiliser de
la même manière. Ce système est appelé \gls{pinning}.\newline

En outre, les nix flakes est un élément essentiel à Nix et est une
technologie que j'ai massivement utilisée lors de mon stage et qui est
utilisée dans de nombreux systèmes tels que NixOS-Compose par
exemple.\newline

\textbf{Exemple d'environnement nix}\newline

\begin{figure}[h]
\centering
\includegraphics[width=0.8\textwidth,height=0.8\textheight,keepaspectratio]{images/flakebasenix.png}
\caption{Script nix de creation d'environnement latex}
\end{figure}

Voici un exemple de création d'environnement isole Nix en utilisant les
flakes nix. Ce script ne marche que sur les architectures x86\_64-linux,
car il ne récupère les dépendances que de ce système d'exploitation ci.
Ce script rajoute dans la PATH du terminal en cours les applications
mise dans les buildInputs, c'est-à-dire dans ce cas pandoc, rubber et
biber. À la fin de cette session, le PATH sera remis à défaut. Pour
l'exécuter, il faut effectuer la commande \texttt{nix\ develop\ .} ou
``.'' est le chemin vers le flake. C'est ce genre de configuration que
j'ai été amené à utiliser et à créer afin d'avoir un environnement et un
résultat reproductible.\newline

\hypertarget{nixos}{%
\subsubsection{NixOS}\label{nixos}}

NixOS est une distribution Linux entièrement basé sur Nix. Il utilise
une approche déclarative pour effectuer la configuration système.
L'intégralité de la configuration est définie par le biais du fichier
configuration.nix. C'est un principe très agréable, car cela permet de
très simplement stocker et versionné la configuration du système afin de
pouvoir par exemple la réutiliser dans une architecture
similaire.\newline

NixOS utilise Nix pour s'occuper de la gestion des paquets. Donc, chaque
paquet est traité manière fonctionnelle. NixOS suit un modèle de mise à
jour sous le nom de \emph{Rolling Release}. Cela consiste à fournir des
mises à jour de manière incrémentale et régulière. Dans le cas de NixOS,
tous les 6 mois. Enfin, le système d'exploitation stocke la
configuration système après chaque changement, permettant de retourner à
tout moment à une configuration précédente en cas de problème.\newline

Pour résumer, NixOS est un Système d'Exploitation innovant et sûr, je
suis ravie d'avoir réalisé l'intégralité de mon stage dans cet
environnement, sur une machine dédié. Cette utilisation intensive de cet
OS m'a permis de développer des compétences système importantes. Le
Système d'Exploitation NixOS a été pour moi une très belle
surprise.\newline

Cependant, il n'est évidemment pas parfait. NixOs utilisant un système
de \emph{Rolling Release} semestriel, il faut souvent réparer la
configuration du système qui s'est vu être modifié par la mise à jour.
De plus, le store Nix propose beaucoup d'atout, mais n'est pas très
efficace quand il faut mettre à jour des paquets régulièrement, comme
Visual Studio Code par exemple. Le store étant immuable, il faut donc
forcer la configuration à utiliser une source plus récente si l'on veut
une version stable de l'application utilisée.\newline

\hypertarget{nixpkgs-et-nur-kapack}{%
\subsubsection{Nixpkgs et Nur-Kapack}\label{nixpkgs-et-nur-kapack}}

Nixpkgs et NUR (Nix User Repository) sont des dépôts de paquets Nix. Ils
sont utilisés massivement le gestionnaire de paquet, en tant que
collection de paquet et logiciel installable par les utilisateurs
possédant Nix. Durant mon stage, j'ai eu la possibilité de rajouter des
paquets dans certain de ces dépôts, afin qu'il soit utilisable par la
communauté Nix.\newline

\textbf{Nixpkgs}\newline

Nixpkgs est le dépôt principal de paquet Nix, il est automatiquement
référencé en tan que tel dans une machine NixOS. Il contient l'un des
plus grands nombres de paquets pour un package manager. plus de 80 000.
Ces paquets peuvent être des outils de développement, des bibliothèques,
des applications, etc. Les paquets disponibles sont ajoutés et maintenus
par la communauté et sont constamment mis à jour afin d'assurer que les
logiciels soient toujours dans une version correcte.\newline

\textbf{NUR}\newline

Nur est un dépôt de paquet supplémentaire à Nix, il est maintenu par des
utilisateurs ou des membres de la communauté. Contrairement à Nixpkgs,
tout le monde peut déposer des paquets dans Nur. Grâce à ce dépôt, il
est donc possible de partager des paquets spécifiques et de les rendre
disponible à la communauté. Nur, est donc une alternative qui permet de
compléter Nixpkgs.\newline

Le fonctionnement de ces outils dépend de la collaboration de la
communauté. Cette collaboration permet à Nix de posséder le plus grand
nombre de paquets disponible dans un gestionnaire de paquet. Et ce de
manière fonctionnel. C'est un élément essentiel de la réussite de Nix et
NixOS.\newline

Durant ce stage, j'ai rajouté des paquets dans des dépôts, afin de les
rendre utilisable par la communauté. Notamment sur le dépôt Nur-kapack
un sous dépôt de NUR, créé par l'équipe DATAMOVE pour y stocké les
paquets important pour la recherche au laboratoire. J'ai eu l'occasion
de comprendre son fonctionnement, tester certain des paquets et donc y
rajouter des fonctionnalités et des paquets.

\newpage

\hypertarget{les-outils-uxe0-disposition-pour-la-recherche}{%
\subsection{Les outils à disposition pour la
recherche}\label{les-outils-uxe0-disposition-pour-la-recherche}}

Test de fonctionnement.

\newpage

\hypertarget{loutils-nixos-compose}{%
\subsection{L'outils NixOS-Compose}\label{loutils-nixos-compose}}

\newpage

\hypertarget{duxe9veloppement-de-composition-nixos-compose}{%
\subsection{Développement de Composition
NixOS-Compose}\label{duxe9veloppement-de-composition-nixos-compose}}

\hypertarget{fonctionnement-et-composition-simple}{%
\subsubsection{Fonctionnement et Composition
Simple}\label{fonctionnement-et-composition-simple}}

\hypertarget{workflow}{%
\subsubsection{Workflow}\label{workflow}}

\hypertarget{grid5000}{%
\subsubsection{Grid5000}\label{grid5000}}

\hypertarget{file-systems}{%
\subsubsection{File Systems}\label{file-systems}}

\hypertarget{mes-contributions-au-projet}{%
\subsubsection{Mes contributions au
projet}\label{mes-contributions-au-projet}}

\newpage

\hypertarget{perspective-du-projet-de-nixos-compose}{%
\subsection{Perspective du projet de
NixOS-Compose}\label{perspective-du-projet-de-nixos-compose}}

\newpage

\hypertarget{conclusion}{%
\section{Conclusion}\label{conclusion}}

\hypertarget{bilan-personnel}{%
\subsection{Bilan personnel}\label{bilan-personnel}}

\hypertarget{bilan-professionnel}{%
\subsection{Bilan professionnel}\label{bilan-professionnel}}

\newpage

\hypertarget{annexe}{%
\section{Annexe}\label{annexe}}

\printbibliography

\printglossaries

\begin{titlingpage}
\newpage
\makefooter
\end{titlingpage}

\end{document}
